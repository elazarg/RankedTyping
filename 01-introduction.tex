\section{Introduction}

Path-insensitive analyses are fast and predictable, but they routinely \emph{collapse} information at control-flow joins.
A common outcome is that a variable alternately assigned an integer and a string is reported as a single top-like ``unknown,'' and every downstream use inherits that imprecision.
The analysis remains sound as an over-approximation, but it is hard to act on: we learn \emph{that} precision was lost, not \emph{where} or \emph{how much}.

We propose an \emph{unfolded} view of the unknown: a \emph{graded} algebra of types that makes loss of precision explicit and compositional.
Base types (\(\Int,\String,\Bool\)) and \(\Any\) are annotated with a nonnegative \emph{grade} \(i\), written \(X^{i}\) and \(\Any^{i}\).
Joins are conservative: after promotion to a common grade \(r\), merging incompatible bases at that grade yields \(\Any^{r}\) (no bump).
Grades increase only when the analysis \emph{commits to a base after imprecision}:
the subtyping step \(\Any^{i} \le X^{\,i+1}\) captures the coercion (assumption) that reinterprets an imprecise value as a concrete base, and increments the grade.
Operational semantics are unchanged; grades live purely in the analysis.

Two consequences follow.
First, graded types record \emph{how much} imprecision was assumed and (via site IDs) \emph{where} assumptions were forced.
Second, although the global order has infinite height, each program inhabits a finite slice:
grade can go up only where a base is demanded after a loss of correlation, and each site is idempotent in the abstract interpretation.
A standard fixed-point computation therefore converges \emph{without} bespoke widenings.

\paragraph{Motivating example.}
Consider a simple, real-world pattern (pseudo-TypeScript):
\begin{quote}\ttfamily
(1) function process(data) \{\\
(2) \ \ if (useCache) \{ data = getCachedString(); \}\\
(3) \ \ else \{ data = computeNumber(); \}\\
(4) \ \ return data.length;\\
(5) \}
\end{quote}
A path-insensitive analysis collapses the join at lines~(2)--(3) and reports ``type is unknown'' at line~(4), leaving the developer to chase the entire dataflow.
Our analysis instead reports \emph{where} and \emph{how much} precision was lost:
\emph{(i)} at the join (lines~2–3) we get $data:\Any^{0}$ (no assumption yet); 
\emph{(ii)} at the use (line~4), the property access forces a \emph{downcast} to the string base, bumping the grade to $data:\String^{1}$; 
\emph{(iii)} the diagnostic points to the join site as the source of imprecision and the use site as the forced assumption.
This graded account is strictly more actionable than a single top-like ``unknown.''

\paragraph{Unfolded gradual typing.}
Conceptually, this is an \emph{unfolded} form of gradual typing: instead of a single unknown \(?\), we expose a stratified family \(\Any^{0},\Any^{1},\ldots\) that records how many assumptions were needed.
Standard gradual typing is recovered by \emph{collapsing} grades (erasing \(X^{i}\mapsto X\), \(\Any^{i}\mapsto ?\)) and adding the usual consistency/cast machinery.
We keep the core static and cast-free; a small extension adds grade-aware casts and a collapsed, consistency-based mode.

\paragraph{Contributions.}
\begin{itemize}
  \item A minimal graded type algebra that exposes path-insensitive precision loss as a natural-number index on types. Joins promote to a common grade and return \(\Any^{r}\) for heterogeneous bases; grades increase only at \emph{coercions} \(\Any^{i}\le X^{\,i+1}\).
  \item A conventional abstract interpreter for a while-language over this domain, with monotone transfer functions and per-program convergence \emph{without} widening (per-site idempotence).
  \item Properties and diagnostics: per-program finiteness; a bounded-blame principle via ghost culprit sets that localizes assumptions to a small set of program sites; and a clear account of what the analysis guarantees (and does not).
\end{itemize}

\paragraph{Utility.}
The immediate use case is diagnostic: instead of reporting a bare ``\(\Any\),'' the analyzer reports \(\Any^{k}\) and points to the joins/uses that forced \(k\) assumptions.
Because grade growth is tied to identifiable choke points where a base is demanded, remediation focuses on a bounded set of locations.
The scheme is unobtrusive: no runtime checks, no language changes, and no change to concrete semantics.
