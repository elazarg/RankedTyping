
\section{Conclusion}
\label{sec:conclusion}

We have presented a static analysis framework tailored to a practical subset of Python, enabling precise liveness, type, and pointer analyses to identify the minimal set of variables that must be preserved across checkpoints in iterative programs. By combining these analyses in a reduced product domain, our approach significantly reduces checkpoint size without sacrificing correctness, even in a highly dynamic language. Experiments on realistic workloads demonstrate that this method achieves substantial memory savings with negligible runtime overhead. Beyond checkpointing, the framework offers a foundation for other transformations that require fine-grained reasoning about program state in Python.

\subsection{Future Directions}

While our work demonstrates that classical static analyses can enable practical checkpoint minimization for a constrained class of Python programs, several directions remain for relaxation and extension.

\paragraph{Relaxing language restrictions.} Our current scope deliberately excludes dynamic code evaluation, reflection and unannotated generics. Relaxing these restrictions would broaden applicability but demands additional analysis. For instance, supporting limited forms of \texttt{getattr} and higher‑order functions could be achieved by incorporating string analysis and call‑graph construction techniques used in prior work on dynamic languages. Likewise, modelling NumPy view aliasing would require a more precise alias analysis for array slices and broadcasting operations.

\paragraph{Tracking dimensions and shapes.} At present the type domain treats all \texttt{numpy.ndarray} objects as homogeneous containers of a single element type. Extending the type system to track array dimensionality and shapes could allow the analysis to reason about, for example, which slice of a tensor is modified in a convolutional kernel, and thereby reduce checkpoint size further. Shape analysis techniques developed for scientific computing languages could be integrated into our type lattice.

\paragraph{Pluggable analyses for other transformations.} The reduced product framework can host additional domains beyond liveness, type and pointer analyses. For example, a numeric range analysis could determine when certain values remain within safe bounds and avoid checkpointing them altogether. A taint analysis could inform selective encryption of sensitive state. Similarly, incorporating loop dependence analysis might allow speculative reordering or batching of checkpoint writes.

\paragraph{Alternative instrumentation strategies.} Our checkpointing instrumentation is tailored to simple for‑loops. Future work could explore automated refactoring of code to expose checkpoint sites in more complex control structures, such as recursion or nested comprehensions. It may also be possible to synthesize checkpointing code in a less intrusive form (e.g., via decorators or context managers) while preserving transparency for developers.

\paragraph{Integrating with performance‑oriented runtimes.} Finally, it would be valuable to evaluate our analyses in the context of alternative Python implementations (e.g., PyPy, Numba or a JIT‑enabled interpreter) and to study how checkpointing interacts with just‑in‑time compilation and garbage collection. Such integration could open the door to broader optimizations, including incremental state re‑materialization and adaptive checkpoint scheduling based on runtime profiling.
