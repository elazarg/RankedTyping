\begin{abstract}
Path-insensitive analyses are cheap but routinely lose precision at joins, collapsing heterogeneous flows to an opaque “unknown.”
We make that loss explicit and quantifiable.
The key insight is a graded algebra of types where base types and $\Any$ carry a nonnegative grade $i$; grades increase \emph{only} when a use forces a base after imprecision (the subtyping step $\Any^{i}\!\le\!X^{i+1}$), while heterogeneous joins are conservative ($X^{r}\sqcup Y^{r}=\Any^{r}$ after promotion).
Instantiated in a conventional abstract interpreter, grades act as a small, compositional signal of \emph{how much} and \emph{where} assumptions were forced.

We instantiate this graded domain in a standard abstract interpreter for a while-language. 
Transfer functions are monotone; joins are associative, commutative, and idempotent. 
Although the global lattice has infinite height, each program inhabits a finite slice, since grades can increase only at finitely many sites. 
Fixed-point iteration therefore converges without widening. 
The analysis is a sound may-analysis: it does not eliminate runtime type errors, but it records and localizes imprecision. 
A lightweight ghost-state instrumentation yields \emph{bounded blame}: any runtime type fault can be traced to a small, computed set of program sites.

Conceptually, this is an ``unfolded'' view of gradual typing. 
Standard gradual typing is recovered by collapsing grades ($X^{i}\mapsto X$, $\Any^{i}\mapsto ?$) and reintroducing consistency and casts. 
We also sketch conservative extensions that reuse the same algebra. 
The core requires no language changes and no runtime checks.
\end{abstract}
